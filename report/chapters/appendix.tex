\chapter{Appendix}

\section{Game Theory}
In this section, we present examples and counterexamples of games with various properties.
These should complete our discussion from \autoref{chap:game-theory} on different equilibria and the relationships between them.

Most of our equilibria involve some kind of round-elimination.
To better visialize the elimination process, we highlight stragy profiles eliminated in different rounds with a diffrent background shade.
Profiles that are eliminated in the very first round are highlighted in the \colorbox{gray!80}{darkest} background.
Every subsequent round of elimination is highlighted with a \colorbox{gray!60}{lighter} and \colorbox{gray!40}{lighter} background.
The last round of elimination is highlighted wih the \colorbox{gray!20}{lightest} backrgoud.
Profiles that survive infinitely many rounds are left without highlighting.

\begin{table}
	\caption{
		IR $\ne$ PTBRE.
		Profiles (B, B), (B, C), (C, B), and (C, C) are individually rational but only (B, B) is a perfectly transparent best profile equilibrium.
	}
	\label{tab:ir-ne-ptbre}
	\centering
	\begin{tabular}{|c|c|c|c|}
		\hline
			& A		& B	   & C	  \\
		\hline
		A 		&\cellcolor{gray!70} 0, 0 &\cellcolor{gray!70} 1, 2 &\cellcolor{gray!70} 3, 4 \\
		\hline
		B		&\cellcolor{gray!70} 2, 1 &\cellcolor{gray!20} 5, 5 &\cellcolor{gray!20} 6, 7 \\
		\hline
		C		&\cellcolor{gray!70} 4, 3 &\cellcolor{gray!20} 7, 6 &\cellcolor{gray!00} 8, 8 \\
		\hline
	\end{tabular}
\end{table}

\begin{table}
	\caption{
		PTBRE $\ne$ PTE.
		Profiles (B, B) and (C, C) are perfectly transparent best response equilibiria but only (B, B) is a perfectly transparent equilibrium.
	}
	\label{tab:ptbre-ne-pte}
	\centering
	\begin{tabular}{|c|c|c|c|}
		\hline
			& A		& B	   & C	  \\
		\hline
		A 		&\cellcolor{gray!70} 0, 0 &\cellcolor{gray!70} 1, 2 &\cellcolor{gray!70} 4, 5 \\
		\hline
		B		&\cellcolor{gray!70} 2, 1 &\cellcolor{gray!00} 8, 8 &\cellcolor{gray!70} 3, 6 \\
		\hline
		C		&\cellcolor{gray!70} 5, 4 &\cellcolor{gray!70} 6, 3 &\cellcolor{gray!20} 7, 7 \\
		\hline
	\end{tabular}
\end{table}

\begin{table}
	\caption{
		PTE $\ne$ PTBPE.
		Strategy profile (B, B) is a PTE.
		Profile (B, C) is a perfectly transparent row-best profile and (C, B) is a perfectly transparent column-best profile, so there is no PTBPE.
	}
	\label{tab:pte-ne-ptbpe}
	\centering
	\begin{tabular}{|c|c|c|c|}
		\hline
			& A		& B	   & C	  \\
		\hline
		A 		&\cellcolor{gray!20} 7, 7 &\cellcolor{gray!70} 6, 4 &\cellcolor{gray!70} 5, 3 \\
		\hline
		B		&\cellcolor{gray!70} 4, 6 &\cellcolor{gray!00} 8, 8 &\cellcolor{gray!70} 9, 1 \\
		\hline
		C		&\cellcolor{gray!70} 3, 5 &\cellcolor{gray!70} 1, 9 &\cellcolor{gray!70} 2, 2 \\
		\hline
	\end{tabular}
\end{table}

\begin{table}
	\caption{
		PTBPE $\ne$ MR.
		In the first copy of the table, preemptively eliminated strategies (used for computing the PTBPE) are highlighted.
		In the second copy, minimax dominated strategies (i.e., those that are not minimax rationalizable) are higlighted.
		Strategy profile (B, F) is a PTBPE, and the only minimax rationalizable profile is (C, F).
	}
	\label{tab:ptbpe-ne-minimax}
	\centering
	\begin{tabular}{|c|c|c|c|}
		\hline
			& D		& E	   & F	  \\
		\hline
		A 		&\cellcolor{gray!80} 1, 1 &\cellcolor{gray!80} 2, 2 &\cellcolor{gray!80} 3, 4 \\
		\hline
		B		&\cellcolor{gray!80} 4, 5 &\cellcolor{gray!40} 6, 8 &\cellcolor{gray!00} 7, 9 \\
		\hline
		C		&\cellcolor{gray!60} 5, 6 &\cellcolor{gray!80} 8, 3 &\cellcolor{gray!60} 9, 7 \\
		\hline
	\end{tabular}
	\hspace{1em}
	\begin{tabular}{|c|c|c|c|}
		\hline
			& D		& E	   & F	  \\
		\hline
		A 		&\cellcolor{gray!80} 1, 1 &\cellcolor{gray!80} 2, 2 &\cellcolor{gray!80} 3, 4 \\
		\hline
		B		&\cellcolor{gray!60} 4, 5 &\cellcolor{gray!40} 6, 8 &\cellcolor{gray!40} 7, 9 \\
		\hline
		C		&\cellcolor{gray!60} 5, 6 &\cellcolor{gray!20} 8, 3 &\cellcolor{gray!00} 9, 7 \\
		\hline
	\end{tabular}
\end{table}


\begin{table}
	\caption{
		PTBPE and MR coincide.
		In this game, minimax elimination and the perfectly transparent preemption eliminate exactly the same profiles in each round.
		Profile (C, C) is both a perfectly transparent best profile equilibrium, and minimax rationalizable.
	}
	\label{tab:ptbpe-eq-minimax}
	\centering
	\begin{tabular}{|c|c|c|c|}
		\hline
			& A		& B	   & C	  \\
		\hline
		A 		&\cellcolor{gray!70} 0, 0 &\cellcolor{gray!70} 1, 2 &\cellcolor{gray!70} 3, 4 \\
		\hline
		B		&\cellcolor{gray!70} 2, 1 &\cellcolor{gray!20} 5, 5 &\cellcolor{gray!20} 6, 7 \\
		\hline
		C		&\cellcolor{gray!70} 4, 3 &\cellcolor{gray!20} 7, 6 &\cellcolor{gray!00} 8, 8 \\
		\hline
	\end{tabular}
\end{table}


\begin{table}
	\caption{
		PTBRE does not exist.
		The perfectly transparent best responses to A, B, C, D are B, D, B, C, respectively (for both row and column player, as the game is symmetric).
	}
	\label{tab:no-ptbre}
	\centering
	\begin{tabular}{|c|c|c|c|c|}
		\hline
			& A		& B	   & C	 & D	 \\
		\hline
		A 		&\cellcolor{gray!70}  1,  1 &\cellcolor{gray!20} 13,  5 &\cellcolor{gray!70}  6,  2 &\cellcolor{gray!70}  9,  0 \\
		\hline
		B		&\cellcolor{gray!20}  5, 13 &\cellcolor{gray!20}  4,  4 &\cellcolor{gray!20} 15,  7 &\cellcolor{gray!20}  8, 14 \\
		\hline
		C		&\cellcolor{gray!70}  2,  6 &\cellcolor{gray!20}  7, 15 &\cellcolor{gray!70}  3,  3 &\cellcolor{gray!20} 12, 10 \\
		\hline
		D		&\cellcolor{gray!70}  0,  9 &\cellcolor{gray!20} 14,  8 &\cellcolor{gray!20} 10, 12 &\cellcolor{gray!20} 11, 11 \\
		\hline
	\end{tabular}
\end{table}
