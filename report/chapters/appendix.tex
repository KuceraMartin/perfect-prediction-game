\chapter{Appendix}

\section{Game Theory}
In this section of the Appendix, we present examples and counterexamples of games with various properties.
These should complete our discussion from \autoref{chap:game-theory} on different equilibria and the relationships between them.

Most of our equilibria involve some sort of round elimination.
To better visualize the elimination process, we highlight strategy profiles eliminated in different rounds with a different background shade.
Profiles that are eliminated in the very first round are highlighted in the \colorbox{gray!80}{darkest} background.
Every subsequent round of elimination is highlighted with a \colorbox{gray!60}{lighter} and \colorbox{gray!40}{lighter} background.
The last round of elimination is highlighted with the \colorbox{gray!20}{lightest} background.
Profiles that survive infinitely many rounds are left without highlighting.

\begin{table}
	\caption{
		PTOPE\textsuperscript{sym} $\not\subseteq$ sPTOPE\textsuperscript{sym}.
		Profiles (B, B) and (C, C) are PTOPE, but only (C, C) is a strict PTOPE.
	}
	\label{tab:ptope-not-sub-sptope}
	\centering
	\begin{tabular}{|c|c|c|c|}
		\hline
			& A		& B	   & C	  \\
		\hline
		A 		&\cellcolor{gray!00} 4, 4 &\cellcolor{gray!00} 7, 2 &\cellcolor{gray!00} 2, 2 \\
		\hline
		B		&\cellcolor{gray!00} 2, 7 &\cellcolor{gray!00} 7, 7 &\cellcolor{gray!70} 6, 0 \\
		\hline
		C		&\cellcolor{gray!00} 2, 2 &\cellcolor{gray!70} 0, 6 &\cellcolor{gray!00} 7, 7 \\
		\hline
	\end{tabular}
\end{table}

\begin{table}
	\caption{
		PTBRE\textsuperscript{sym} $\not\subseteq$ sPTBRE\textsuperscript{sym}.
		Profiles (A, A), (B, B), and (C, C) are PTBRE, but only (A, A) and (B, B) are strict PTBRE.
	}
	\label{tab:ptbre-not-sub-sptbre}
	\centering
	\begin{tabular}{|c|c|c|c|}
		\hline
			& A		& B	   & C	  \\
		\hline
		A 		&\cellcolor{gray!20} 6, 6 &\cellcolor{gray!70} 8, 1 &\cellcolor{gray!70} 4, 5 \\
		\hline
		B		&\cellcolor{gray!70} 1, 8 &\cellcolor{gray!00} 7, 7 &\cellcolor{gray!45} 5, 5 \\
		\hline
		C		&\cellcolor{gray!70} 5, 4 &\cellcolor{gray!45} 5, 5 &\cellcolor{gray!45} 5, 5 \\
		\hline
	\end{tabular}
\end{table}

\begin{table}
	\caption{
		sPTOPE\textsuperscript{sym} $\not\subseteq$ PTBPE\textsuperscript{sym}.
		Profile (C, C) is a strict PTOPE but there is no PTBPE because the only perfectly transparent row-best profile is (B, C) and the only column-best profile is (C, B).
	}
	\label{tab:sptope-not-sub-ptbpe}
	\centering
	\begin{tabular}{|c|c|c|c|}
		\hline
			& A		& B	   & C	  \\
		\hline
		A 		&\cellcolor{gray!70} 0, 0 &\cellcolor{gray!70} 1, 2 &\cellcolor{gray!20} 6, 3 \\
		\hline
		B		&\cellcolor{gray!70} 2, 1 &\cellcolor{gray!20} 4, 4 &\cellcolor{gray!20} 8, 5 \\
		\hline
		C		&\cellcolor{gray!20} 3, 6 &\cellcolor{gray!20} 5, 8 &\cellcolor{gray!00} 7, 7 \\
		\hline
	\end{tabular}
\end{table}

\begin{table}
	\caption{
		\\sPTBPE\textsuperscript{sym} $\not\subseteq$ PTOPE\textsuperscript{sym}.
		Profile (C, C) is a strict PTBPE but there is no PTOPE because the only perfectly transparent row-optimal profile is (C, B) and the only column-optimal profile is (B, C).\vspace{5pt}\\
		sPTBRE\textsuperscript{sym} $\not\subseteq$ PTOPE\textsuperscript{sym}.
		Profile (C, C) is a strict PTBRE but there is no PTOPE.
	}
	\label{tab:sptbpe-not-sub-ptope}
	\label{tab:sptbre-not-sub-ptope}
	\centering
	\begin{tabular}{|c|c|c|c|}
		\hline
			& A		& B	   & C	  \\
		\hline
		A 		&\cellcolor{gray!70} 0, 0 &\cellcolor{gray!70} 1, 2 &\cellcolor{gray!70} 3, 4 \\
		\hline
		B		&\cellcolor{gray!70} 2, 1 &\cellcolor{gray!20} 5, 5 &\cellcolor{gray!20} 6, 8 \\
		\hline
		C		&\cellcolor{gray!70} 4, 3 &\cellcolor{gray!20} 8, 6 &\cellcolor{gray!00} 7, 7 \\
		\hline
	\end{tabular}
\end{table}

\begin{table}
	\caption{
		\\PTOPE\textsuperscript{sym} $\not\subseteq$ sPTBRE\textsuperscript{sym}.
		Profiles (A, A), (A, C), (B, B), and (C, A) are PTOPE but only (B, B) is a strict PTBRE.\vspace{5pt}\\
		PTBPE\textsuperscript{sym} $\not\subseteq$ sPTBRE\textsuperscript{sym}.
		Profiles (A, A), (A, C), (B, B), and (C, A) are PTBPE but only (B, B) is a strict PTBRE.
	}
	\label{tab:PTOPE-not-sub-sptbre}
	\label{tab:ptbpe-not-sub-sptbre}
	\centering
	\begin{tabular}{|c|c|c|c|}
		\hline
			& A		& B	   & C	  \\
		\hline
		A 		&\cellcolor{gray!00} 8, 8 &\cellcolor{gray!20} 7, 8 &\cellcolor{gray!00} 8, 8 \\
		\hline
		B		&\cellcolor{gray!20} 8, 7 &\cellcolor{gray!00} 8, 8 &\cellcolor{gray!70} 0, 1 \\
		\hline
		C		&\cellcolor{gray!00} 8, 8 &\cellcolor{gray!70} 1, 0 &\cellcolor{gray!70} 1, 1 \\
		\hline
	\end{tabular}
\end{table}

\begin{table}
	\caption{
		sPTBRE\textsuperscript{sym} $\not\subseteq$ PTBPE\textsuperscript{sym}.
		Profiles (B, C) and (C, B) are PTBRE but there is no PTBPE because the only perfectly transparent row-best profile is (C, B) and the only column-best profile is (B, C).
	}
	\label{tab:ptbre-not-sub-ptbpe}
	\centering
	\begin{tabular}{|c|c|c|c|}
		\hline
			& A		& B	   & C	  \\
		\hline
		A 		&\cellcolor{gray!80} 0, 0 &\cellcolor{gray!80} 1, 2 &\cellcolor{gray!80} 3, 4 \\
		\hline
		B		&\cellcolor{gray!80} 2, 1 &\cellcolor{gray!60} 5, 5 &\cellcolor{gray!40} 7, 8 \\
		\hline
		C		&\cellcolor{gray!80} 4, 3 &\cellcolor{gray!40} 8, 7 &\cellcolor{gray!20} 6, 6 \\
		\hline
	\end{tabular}
\end{table}

\begin{table}
	\caption{
		\\IR\textsuperscript{sym} $\not\subseteq$ PTBRE\textsuperscript{sym}.
		Profiles (B, B), (B, C), (C, B), and (C, C) are individually rational but only (C, C) is a PTBRE.\vspace{5pt}\\
		IR\textsuperscript{sym} $\not\subseteq$ PTE\textsuperscript{sym}.
		Profiles (B, B), (B, C), (C, B), and (C, C) are individually rational but only (C, C) is a PTE.\vspace{5pt}\\
		IR\textsuperscript{sym} $\not\subseteq$ PTBPE\textsuperscript{sym}.
		Profiles (B, B), (B, C), (C, B), and (C, C) are individually rational but only (C, C) is a PTBPE.
	}
	\label{tab:ir-ne-ptbre}
	\label{tab:ir-not-sub-pte}
	\label{tab:ir-not-sub-ptbpe}
	\centering
	\begin{tabular}{|c|c|c|c|}
		\hline
			& A		& B	   & C	  \\
		\hline
		A 		&\cellcolor{gray!70} 0, 0 &\cellcolor{gray!70} 1, 2 &\cellcolor{gray!70} 3, 4 \\
		\hline
		B		&\cellcolor{gray!70} 2, 1 &\cellcolor{gray!20} 5, 5 &\cellcolor{gray!20} 6, 7 \\
		\hline
		C		&\cellcolor{gray!70} 4, 3 &\cellcolor{gray!20} 7, 6 &\cellcolor{gray!00} 8, 8 \\
		\hline
	\end{tabular}
\end{table}

\begin{table}
	\caption{
		sPTBRE\textsuperscript{sym} $\not\subseteq$ PTE\textsuperscript{sym}.
		Profiles (B, B) and (C, C) are perfectly transparent best response equilibiria but only (B, B) is a perfectly transparent equilibrium.
	}
	\label{tab:ptbre-ne-pte}
	\centering
	\begin{tabular}{|c|c|c|c|}
		\hline
			& A		& B	   & C	  \\
		\hline
		A 		&\cellcolor{gray!70} 0, 0 &\cellcolor{gray!70} 1, 2 &\cellcolor{gray!70} 4, 5 \\
		\hline
		B		&\cellcolor{gray!70} 2, 1 &\cellcolor{gray!00} 8, 8 &\cellcolor{gray!70} 3, 6 \\
		\hline
		C		&\cellcolor{gray!70} 5, 4 &\cellcolor{gray!70} 6, 3 &\cellcolor{gray!20} 7, 7 \\
		\hline
	\end{tabular}
\end{table}

\begin{table}
	\caption{
		PTE\textsuperscript{sym} $\not\subseteq$ PTOPE\textsuperscript{sym}.
		Strategy profile (B, B) is a PTE.
		Profile (B, C) is a perfectly transparent row-optimal profile and (C, B) is a perfectly transparent column-optimal profile, so there is no PTOPE.
	}
	\label{tab:pte-ne-PTOPE}
	\centering
	\begin{tabular}{|c|c|c|c|}
		\hline
			& A		& B	   & C	  \\
		\hline
		A 		&\cellcolor{gray!20} 7, 7 &\cellcolor{gray!70} 6, 4 &\cellcolor{gray!70} 5, 3 \\
		\hline
		B		&\cellcolor{gray!70} 4, 6 &\cellcolor{gray!00} 8, 8 &\cellcolor{gray!70} 9, 1 \\
		\hline
		C		&\cellcolor{gray!70} 3, 5 &\cellcolor{gray!70} 1, 9 &\cellcolor{gray!70} 2, 2 \\
		\hline
	\end{tabular}
\end{table}

\begin{table}
	\caption{PTOPE $\not\subseteq$ MR.
		In the first copy of the table, preemptively eliminated strategies (used for computing the PTOPE) are highlighted.
		In the second copy, minimax-dominated strategies (i.e., those that are not minimax rationalizable) are highlighted.
		Strategy profile (B, F) is a PTOPE, and the only minimax rationalizable profile is (C, F).
	}
	\label{tab:PTOPE-ne-minimax}
	\centering
	\begin{tabular}{|c|c|c|c|}
		\hline
			& D		& E	   & F	  \\
		\hline
		A 		&\cellcolor{gray!80} 1, 1 &\cellcolor{gray!80} 2, 2 &\cellcolor{gray!80} 3, 4 \\
		\hline
		B		&\cellcolor{gray!80} 4, 5 &\cellcolor{gray!40} 6, 8 &\cellcolor{gray!00} 7, 9 \\
		\hline
		C		&\cellcolor{gray!60} 5, 6 &\cellcolor{gray!80} 8, 3 &\cellcolor{gray!60} 9, 7 \\
		\hline
	\end{tabular}
	\hspace{1em}
	\begin{tabular}{|c|c|c|c|}
		\hline
			& D		& E	   & F	  \\
		\hline
		A 		&\cellcolor{gray!80} 1, 1 &\cellcolor{gray!80} 2, 2 &\cellcolor{gray!80} 3, 4 \\
		\hline
		B		&\cellcolor{gray!60} 4, 5 &\cellcolor{gray!40} 6, 8 &\cellcolor{gray!40} 7, 9 \\
		\hline
		C		&\cellcolor{gray!60} 5, 6 &\cellcolor{gray!20} 8, 3 &\cellcolor{gray!00} 9, 7 \\
		\hline
	\end{tabular}
\end{table}


\begin{table}
	\caption{
		PTOPE and MR coincide.
		In this game, minimax elimination and the perfectly transparent preemption eliminate exactly the same profiles in each round.
		Profile (C, C) is both a perfectly transparent optimal profile equilibrium and minimax rationalizable.
	}
	\label{tab:PTOPE-eq-minimax}
	\centering
	\begin{tabular}{|c|c|c|c|}
		\hline
			& A		& B	   & C	  \\
		\hline
		A 		&\cellcolor{gray!70} 0, 0 &\cellcolor{gray!70} 1, 2 &\cellcolor{gray!70} 3, 4 \\
		\hline
		B		&\cellcolor{gray!70} 2, 1 &\cellcolor{gray!20} 5, 5 &\cellcolor{gray!20} 6, 7 \\
		\hline
		C		&\cellcolor{gray!70} 4, 3 &\cellcolor{gray!20} 7, 6 &\cellcolor{gray!00} 8, 8 \\
		\hline
	\end{tabular}
\end{table}


\begin{table}
	\caption{
		PTBRE does not exist.
		The perfectly transparent best responses to A, B, C, D are B, D, B, C, respectively (for both row and column player, as the game is symmetric).
	}
	\label{tab:no-ptbre}
	\centering
	\begin{tabular}{|c|c|c|c|c|}
		\hline
			& A		& B	   & C	 & D	 \\
		\hline
		A 		&\cellcolor{gray!70}  1,  1 &\cellcolor{gray!20} 13,  5 &\cellcolor{gray!70}  6,  2 &\cellcolor{gray!70}  9,  0 \\
		\hline
		B		&\cellcolor{gray!20}  5, 13 &\cellcolor{gray!20}  4,  4 &\cellcolor{gray!20} 15,  7 &\cellcolor{gray!20}  8, 14 \\
		\hline
		C		&\cellcolor{gray!70}  2,  6 &\cellcolor{gray!20}  7, 15 &\cellcolor{gray!70}  3,  3 &\cellcolor{gray!20} 12, 10 \\
		\hline
		D		&\cellcolor{gray!70}  0,  9 &\cellcolor{gray!20} 14,  8 &\cellcolor{gray!20} 10, 12 &\cellcolor{gray!20} 11, 11 \\
		\hline
	\end{tabular}
\end{table}

\begin{table}
	\caption{
		sPTBRE\textsuperscript{sym} $\not\subseteq$ MR\textsuperscript{sym}.
		In the first copy of the table, preemptively eliminated strategies (used for computing PTBRE) are highlighted.
		In the second copy, minimax-dominated strategies (i.e., those that are not minimax rationalizable) are highlighted.
		Strategy profiles (B, B) and (C, C) are a PTBRE but only (C, C) is MR.
	}
	\label{tab:sym-ptbre-ne-mr}
	\centering
	\begin{tabular}{|c|c|c|c|}
		\hline
			& A		& B	   & C	  \\
		\hline
		A 		&\cellcolor{gray!70} 0, 0 &\cellcolor{gray!70} 1, 2 &\cellcolor{gray!70} 3, 5 \\
		\hline
		B		&\cellcolor{gray!70} 2, 1 &\cellcolor{gray!20} 6, 6 &\cellcolor{gray!70} 4, 7 \\
		\hline
		C		&\cellcolor{gray!70} 5, 3 &\cellcolor{gray!70} 7, 4 &\cellcolor{gray!00} 8, 8 \\
		\hline
	\end{tabular}
	\hspace{1em}
	\begin{tabular}{|c|c|c|c|}
		\hline
			& A		& B	   & C	  \\
		\hline
		A 		&\cellcolor{gray!70} 0, 0 &\cellcolor{gray!70} 1, 2 &\cellcolor{gray!70} 3, 5 \\
		\hline
		B		&\cellcolor{gray!70} 2, 1 &\cellcolor{gray!20} 6, 6 &\cellcolor{gray!20} 4, 7 \\
		\hline
		C		&\cellcolor{gray!70} 5, 3 &\cellcolor{gray!20} 7, 4 &\cellcolor{gray!00} 8, 8 \\
		\hline
	\end{tabular}
\end{table}

\begin{table}
	\caption{
		MR\textsuperscript{sym} $\not\subset$ IR\textsuperscript{sym}.
		In the first copy of the table, highlighted strategy profiles are not individually rational.
		In the second copy, minimax-dominated strategy profiles are highlighted.
		Strategy profile (B, B) is minimax rationalizable, but it is not individually rational.
	}
	\label{tab:mr-not-sub-ir}
	\centering
	\begin{tabular}{|c|c|c|c|}
		\hline
			& A		& B	   & C	  \\
		\hline
		A 		&\cellcolor{gray!70} 0, 0 &\cellcolor{gray!70} 1, 2 &\cellcolor{gray!70} 3, 5 \\
		\hline
		B		&\cellcolor{gray!70} 2, 1 &\cellcolor{gray!70} 4, 4 &\cellcolor{gray!00} 7, 6 \\
		\hline
		C		&\cellcolor{gray!70} 5, 3 &\cellcolor{gray!00} 6, 7 &\cellcolor{gray!00} 8, 8 \\
		\hline
	\end{tabular}
	\hspace{1em}
	\begin{tabular}{|c|c|c|c|}
		\hline
			& A		& B	   & C	  \\
		\hline
		A 		&\cellcolor{gray!70} 0, 0 &\cellcolor{gray!70} 1, 2 &\cellcolor{gray!70} 3, 5 \\
		\hline
		B		&\cellcolor{gray!70} 2, 1 &\cellcolor{gray!00} 4, 4 &\cellcolor{gray!00} 7, 6 \\
		\hline
		C		&\cellcolor{gray!70} 5, 3 &\cellcolor{gray!00} 6, 7 &\cellcolor{gray!00} 8, 8 \\
		\hline
	\end{tabular}
\end{table}

\begin{table}
	\caption{
		PTE\textsuperscript{sym} $\not\subseteq$ PTBPE\textsuperscript{sym}.
		Profile (B, B) is a PTE, but there is no PTBPE because the only perfectly transparent row-best profile is (C, A) and the only column-best profile is (A, C).
	}
	\label{tab:pte-not-sub-ptbpe}
	\centering
	\begin{tabular}{|c|c|c|c|}
		\hline
			& A		& B	   & C	  \\
		\hline
		A 		&\cellcolor{gray!70} 0, 0 &\cellcolor{gray!70} 1, 2 &\cellcolor{gray!20} 5, 7 \\
		\hline
		B		&\cellcolor{gray!70} 2, 1 &\cellcolor{gray!00} 6, 6 &\cellcolor{gray!70} 3, 8 \\
		\hline
		C		&\cellcolor{gray!20} 7, 5 &\cellcolor{gray!20} 8, 3 &\cellcolor{gray!20} 4, 4 \\
		\hline
	\end{tabular}
\end{table}

\begin{table}
	\caption{
		PTE\textsuperscript{sym} $\not\subseteq$ PTBRE\textsuperscript{sym}.
		Profiles (A, A), (A, C), and (C, A) are PTE, but only (A, A) is a PTBRE.
	}
	\label{tab:ties-pte-not-sub-ptbre}
	\centering
	\begin{tabular}{|c|c|c|c|}
		\hline
			& A		& B	   & C	  \\
		\hline
		A 		&\cellcolor{gray!00} 8, 8 &\cellcolor{gray!70} 4, 0 &\cellcolor{gray!00} 7, 7 \\
		\hline
		B		&\cellcolor{gray!70} 0, 4 &\cellcolor{gray!70} 2, 2 &\cellcolor{gray!70} 1, 2 \\
		\hline
		C		&\cellcolor{gray!00} 7, 7 &\cellcolor{gray!70} 2, 1 &\cellcolor{gray!70} 3, 3 \\
		\hline
	\end{tabular}
\end{table}


\section{Inclusions Overview}
\label{sec:inclusions-overview}
In this section of the Appendix, we summarize all the inclusion claims made in this report.
Inclusions for games with ties are displayed in \autoref{tab:overview-nosym-dup}, inclusions for symmetric games with ties are displayed in \autoref{tab:overview-sym-dup}, inclusions for games without ties are displayed in \autoref{tab:overview-nosym-nodup}, and inclusions for symmetric games without ties are displayed in \autoref{tab:overview-sym-nodup}.
Each table contains one row and one column for every equilibrium.
The respective cell for row $A$ and column $B$ contains a reference to either a proof or a counterexample for the claim \enquote{$A \subseteq B$}.
Every cell referring to a proof (i.e., the claim is true) is highlighted in \colorbox{green!30}{green} and each cell referring to a counterexample (i.e., the claim is false) is highlighted in \colorbox{red!30}{red}.
For some inclusions, we do not present direct proofs or counterexamples but rather these are implied transitively from other inclusions that we prove.
Such inclusions are highlighted in \colorbox{green!15}{light green} if they are true, and in \colorbox{red!15}{light red} if they are false.
Inclusions highlighted in \colorbox{gray!30}{grey} are trivially true by reflexivity.

\begin{sidewaystable}
	\caption{Inclusions overview in non-symmetric games with ties. See \autoref{sec:inclusions-overview} for an explanation.}
	\label{tab:overview-nosym-dup}
	\centering
	\begin{tabular}{|c|c|c|c|c|c|c|c|c|c|}
		\hline
		$\subseteq$  & sPTOPE & PTOPE & sPTBPE & PTBPE & sPTBRE & PTBRE & PTE & IR & MR \\
		\hline
		sPTOPE & \cellcolor{gray!30} & \cellcolor{green!30} \autoref{th:strict-sub-weak} & \cellcolor{red!15} & \cellcolor{red!25} \autoref{tab:sptope-not-sub-ptbpe} & \cellcolor{green!30} \autoref{th:ptope-subset-ptbre} & \cellcolor{green!15} & \cellcolor{green!15} & \cellcolor{green!15} & \cellcolor{red!30} \autoref{tab:PTOPE-ne-minimax} \\
		\hline
		PTOPE & \cellcolor{red!30} \autoref{tab:ptope-not-sub-sptope} & \cellcolor{gray!30} & \cellcolor{red!15} & \cellcolor{red!15} & \cellcolor{red!30} \autoref{tab:PTOPE-not-sub-sptbre} & \cellcolor{green!30} \autoref{th:ptope-subset-ptbre} & \cellcolor{green!30} \autoref{th:ptope-subset-pte} & \cellcolor{green!15} & \cellcolor{red!15} \\
		\hline
		sPTBPE & \cellcolor{red!15} & \cellcolor{red!30} \autoref{tab:sptbpe-not-sub-ptope} & \cellcolor{gray!30} & \cellcolor{green!30} \autoref{th:strict-sub-weak} & \cellcolor{green!30} \autoref{th:ptbpe-subset-ptbre} & \cellcolor{green!15} & \cellcolor{green!15} & \cellcolor{green!15} & \cellcolor{green!15} \\
		\hline
		PTBPE & \cellcolor{red!15} & \cellcolor{red!15} & \cellcolor{red!15} & \cellcolor{gray!30} & \cellcolor{red!30} \autoref{tab:ptbpe-not-sub-sptbre} & \cellcolor{green!30} \autoref{th:ptbpe-subset-ptbre} & \cellcolor{green!30} \autoref{th:ptbpe-subset-pte} & \cellcolor{green!15} & \cellcolor{green!30} \autoref{th:ptbpe-subset-mr} \\
		\hline
		sPTBRE & \cellcolor{red!15} & \cellcolor{red!30} \autoref{tab:sptbre-not-sub-ptope} & \cellcolor{red!15} & \cellcolor{red!30} \autoref{tab:ptbre-not-sub-ptbpe} & \cellcolor{gray!30} & \cellcolor{green!30} \autoref{th:strict-sub-weak} & \cellcolor{red!30} \autoref{tab:ptbre-ne-pte} & \cellcolor{green!15} & \cellcolor{red!30} \autoref{tab:sym-ptbre-ne-mr} \\
		\hline
		PTBRE & \cellcolor{red!15} & \cellcolor{red!15} & \cellcolor{red!15} & \cellcolor{red!15} & \cellcolor{red!30} \autoref{tab:ptbre-not-sub-sptbre} & \cellcolor{gray!30} & \cellcolor{red!15} & \cellcolor{green!30} \autoref{th:ptbre-subset-ir} & \cellcolor{red!15} \\
		\hline
		PTE & \cellcolor{red!15} & \cellcolor{red!30} \autoref{tab:pte-ne-PTOPE} & \cellcolor{red!15} & \cellcolor{red!30} \autoref{tab:pte-not-sub-ptbpe} & \cellcolor{red!15} & \cellcolor{red!30} \autoref{tab:ties-pte-not-sub-ptbre} & \cellcolor{gray!30} & \cellcolor{green!30} \autoref{th:pte-subset-ir} & \cellcolor{red!15} \\
		\hline
		IR & \cellcolor{red!15} & \cellcolor{red!15} & \cellcolor{red!15} & \cellcolor{red!30} \autoref{tab:ir-not-sub-ptbpe} & \cellcolor{red!15} & \cellcolor{red!30} \autoref{tab:ir-ne-ptbre} & \cellcolor{red!30} \autoref{tab:ir-not-sub-pte} & \cellcolor{gray!30} & \cellcolor{red!15} \\
		\hline
		MR & \cellcolor{red!15} & \cellcolor{red!15} & \cellcolor{red!15} & \cellcolor{red!15} & \cellcolor{red!15} & \cellcolor{red!15} & \cellcolor{red!15} & \cellcolor{red!30} \autoref{tab:mr-not-sub-ir} & \cellcolor{gray!30} \\
		\hline
		\end{tabular}
\end{sidewaystable}

\begin{sidewaystable}
	\caption{Inclusions overview in symmetric games with ties. See \autoref{sec:inclusions-overview} for an explanation.}
	\label{tab:overview-sym-dup}
	\hspace{-4cm}
	\begin{tabular}{|c|c|c|c|c|c|c|c|c|c|}
		\hline
		$\subseteq$  & sPTOPE & PTOPE & sPTBPE & PTBPE & sPTBRE & PTBRE & PTE & IR & MR \\
		\hline
		sPTOPE & \cellcolor{gray!30} & \cellcolor{green!30} \autoref{th:strict-sub-weak} & \cellcolor{red!15} & \cellcolor{red!25} \autoref{tab:sptope-not-sub-ptbpe} & \cellcolor{green!30} \autoref{th:ptope-subset-ptbre} & \cellcolor{green!15} & \cellcolor{green!15} & \cellcolor{green!15} & ? \\
		\hline
		PTOPE & \cellcolor{red!30} \autoref{tab:ptope-not-sub-sptope} & \cellcolor{gray!30} & \cellcolor{red!15} & \cellcolor{red!15} & \cellcolor{red!30} \autoref{tab:PTOPE-not-sub-sptbre} & \cellcolor{green!30} \autoref{th:ptope-subset-ptbre} & \cellcolor{green!30} \autoref{th:ptope-subset-pte} & \cellcolor{green!15} & ? \\
		\hline
		sPTBPE & \cellcolor{red!15} & \cellcolor{red!30} \autoref{tab:sptbpe-not-sub-ptope} & \cellcolor{gray!30} & \cellcolor{green!30} \autoref{th:strict-sub-weak} & \cellcolor{green!30} \autoref{th:ptbpe-subset-ptbre} & \cellcolor{green!15} & \cellcolor{green!15} & \cellcolor{green!15} & \cellcolor{green!15} \\
		\hline
		PTBPE & \cellcolor{red!15} & \cellcolor{red!15} & \cellcolor{red!15} & \cellcolor{gray!30} & \cellcolor{red!30} \autoref{tab:ptbpe-not-sub-sptbre} & \cellcolor{green!30} \autoref{th:ptbpe-subset-ptbre} & \cellcolor{green!30} \autoref{th:ptbpe-subset-pte} & \cellcolor{green!15} & \cellcolor{green!30} \autoref{th:ptbpe-subset-mr} \\
		\hline
		sPTBRE & \cellcolor{red!15} & \cellcolor{red!30} \autoref{tab:sptbre-not-sub-ptope} & \cellcolor{red!15} & \cellcolor{red!30} \autoref{tab:ptbre-not-sub-ptbpe} & \cellcolor{gray!30} & \cellcolor{green!30} \autoref{th:strict-sub-weak} & \cellcolor{red!30} \autoref{tab:ptbre-ne-pte} & \cellcolor{green!15} & \cellcolor{red!30} \autoref{tab:sym-ptbre-ne-mr} \\
		\hline
		PTBRE & \cellcolor{red!15} & \cellcolor{red!15} & \cellcolor{red!15} & \cellcolor{red!15} & \cellcolor{red!30} \autoref{tab:ptbre-not-sub-sptbre} & \cellcolor{gray!30} & \cellcolor{red!15} & \cellcolor{green!30} \autoref{th:ptbre-subset-ir} & \cellcolor{red!15} \\
		\hline
		PTE & \cellcolor{red!15} & \cellcolor{red!30} \autoref{tab:pte-ne-PTOPE} & \cellcolor{red!15} & \cellcolor{red!30} \autoref{tab:pte-not-sub-ptbpe} & \cellcolor{red!15} & \cellcolor{red!30} \autoref{tab:ties-pte-not-sub-ptbre} & \cellcolor{gray!30} & \cellcolor{green!30} \autoref{th:pte-subset-ir} & ? \\
		\hline
		IR & \cellcolor{red!15} & \cellcolor{red!15} & \cellcolor{red!15} & \cellcolor{red!30} \autoref{tab:ir-not-sub-ptbpe} & \cellcolor{red!15} & \cellcolor{red!30} \autoref{tab:ir-ne-ptbre} & \cellcolor{red!30} \autoref{tab:ir-not-sub-pte} & \cellcolor{gray!30} & \cellcolor{red!15} \\
		\hline
		MR & \cellcolor{red!15} & \cellcolor{red!15} & \cellcolor{red!15} & \cellcolor{red!15} & \cellcolor{red!15} & \cellcolor{red!15} & \cellcolor{red!15} & \cellcolor{red!30} \autoref{tab:mr-not-sub-ir} & \cellcolor{gray!30} \\
		\hline
		\end{tabular}
\end{sidewaystable}

\begin{sidewaystable}
	\caption{Inclusions overview in non-symmetric games without ties. See \autoref{sec:inclusions-overview} for an explanation.}
	\label{tab:overview-nosym-nodup}
	\begin{tabular}{|c|c|c|c|c|c|c|}
		\hline
		$\subseteq$  & PTOPE & PTBPE & PTBRE & PTE & IR & MR \\
		\hline
		PTOPE & \cellcolor{gray!30} & \cellcolor{red!30} \autoref{tab:sptope-not-sub-ptbpe} & \cellcolor{green!30} \autoref{th:ptope-subset-ptbre} & \cellcolor{green!30} \autoref{th:ptope-subset-pte} & \cellcolor{green!15} & \cellcolor{red!30} \autoref{tab:PTOPE-ne-minimax} \\
		\hline
		PTBPE & \cellcolor{red!30} \autoref{tab:sptbpe-not-sub-ptope} & \cellcolor{gray!30} & \cellcolor{green!30} \autoref{th:ptbpe-subset-ptbre} & \cellcolor{green!30} \autoref{th:ptbpe-subset-pte} & \cellcolor{green!15} & \cellcolor{green!30} \autoref{th:ptbpe-subset-mr} \\
		\hline
		PTBRE & \cellcolor{red!30} \autoref{tab:sptbre-not-sub-ptope} & \cellcolor{red!30} \autoref{tab:ptbre-not-sub-ptbpe} & \cellcolor{gray!30} & \cellcolor{red!30} \autoref{tab:ptbre-ne-pte} & \cellcolor{green!30} \autoref{th:ptbre-subset-ir} & \cellcolor{red!30} \autoref{tab:sym-ptbre-ne-mr} \\
		\hline
		PTE & \cellcolor{red!30} \autoref{tab:pte-ne-PTOPE} & \cellcolor{red!30} \autoref{tab:pte-not-sub-ptbpe} & \cellcolor{green!30} \autoref{th:pte-subset-ptbre} & \cellcolor{gray!30} & \cellcolor{green!30} \autoref{th:pte-subset-ir} & \cellcolor{red!15} \\
		\hline
		IR & \cellcolor{red!15} & \cellcolor{red!30} \autoref{tab:ir-not-sub-ptbpe} & \cellcolor{red!30} \autoref{tab:ir-ne-ptbre} & \cellcolor{red!30} \autoref{tab:ir-not-sub-pte} & \cellcolor{gray!30} & \cellcolor{red!15} \\
		\hline
		MR & \cellcolor{red!15} & \cellcolor{red!15} & \cellcolor{red!15} & \cellcolor{red!15} & \cellcolor{red!30} \autoref{tab:mr-not-sub-ir} & \cellcolor{gray!30} \\
		\hline
		\end{tabular}
\end{sidewaystable}

\begin{sidewaystable}
	\caption{Inclusions overview in symmetric games without ties. See \autoref{sec:inclusions-overview} for an explanation.}
	\label{tab:overview-sym-nodup}
	\begin{tabular}{|c|c|c|c|c|c|c|}
		\hline
		$\subseteq$  & PTOPE & PTBPE & PTBRE & PTE & IR & MR \\
		\hline
		PTOPE & \cellcolor{gray!30} & \cellcolor{red!30} \autoref{tab:sptope-not-sub-ptbpe} & \cellcolor{green!30} \autoref{th:ptope-subset-ptbre} & \cellcolor{green!30} \autoref{th:ptope-subset-pte} & \cellcolor{green!15} & \cellcolor{green!15} \\
		\hline
		PTBPE & \cellcolor{red!30} \autoref{tab:sptbpe-not-sub-ptope} & \cellcolor{gray!30} & \cellcolor{green!30} \autoref{th:ptbpe-subset-ptbre} & \cellcolor{green!30} \autoref{th:ptbpe-subset-pte} & \cellcolor{green!15} & \cellcolor{green!15} \\
		\hline
		PTBRE & \cellcolor{red!30} \autoref{tab:sptbre-not-sub-ptope} & \cellcolor{red!30} \autoref{tab:ptbre-not-sub-ptbpe} & \cellcolor{gray!30} & \cellcolor{red!30} \autoref{tab:ptbre-ne-pte} & \cellcolor{green!30} \autoref{th:ptbre-subset-ir} & \cellcolor{red!30} \autoref{tab:sym-ptbre-ne-mr} \\
		\hline
		PTE & \cellcolor{red!30} \autoref{tab:pte-ne-PTOPE} & \cellcolor{red!30} \autoref{tab:pte-not-sub-ptbpe} & \cellcolor{green!30} \autoref{th:pte-subset-ptbre} & \cellcolor{gray!30} & \cellcolor{green!30} \autoref{th:pte-subset-ir} & \cellcolor{green!30} \autoref{th:pte-sym-subset-mr} \\
		\hline
		IR & \cellcolor{red!15} & \cellcolor{red!30} \autoref{tab:ir-not-sub-ptbpe} & \cellcolor{red!30} \autoref{tab:ir-ne-ptbre} & \cellcolor{red!30} \autoref{tab:ir-not-sub-pte} & \cellcolor{gray!30} & \cellcolor{red!15} \\
		\hline
		MR & \cellcolor{red!15} & \cellcolor{red!15} & \cellcolor{red!15} & \cellcolor{red!15} & \cellcolor{red!30} \autoref{tab:mr-not-sub-ir} & \cellcolor{gray!30} \\
		\hline
		\end{tabular}
\end{sidewaystable}
