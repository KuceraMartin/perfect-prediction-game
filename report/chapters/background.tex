\chapter{Background}
\label{chap:background}

\section{Nashian Game Theory}
Here we introduce some basic notions from algorithmic game theory.
A knowledgable reader may safely skip this section.

We start by defining a game in normal form.
Intuitively, it is a game of $n$ players in which each player $i$ has a set of strategies $S_i$, and the goal is to maximize \textit{utility} by choosing the best strategy $s_i \in S_i$.
A vector of length $n$ in which the $i$\textsuperscript{th} element describes the strategy chosen by player $i$ is called a \textit{strategy profile}.
The utility (somtimes also called payoff) is then determined individually for each player based on the strategy profile.
The game is only played once.
Next, we provide a formal definition.

\begin{definition}[Game in normal form]
  We define a game in normal form as a 3-tuple $G = (P, S, \uu)$ where
  \begin{itemize}
    \item $P = \{1, \dots n\}$ is the set of players,
    \item $S$ = $S_1 \times S_2 \times \dots \times S_n$ is the set of strategy profiles,
    \begin{itemize}
      \item $S_i$ is the set of strategies of player $i \in P$,
    \end{itemize}
    \item $\uu = (u_1, u_2, \dots, u_n)$ is the vector of utility functions,
    \begin{itemize}
      \item $u_i: S \rightarrow \R$ is the utility function for player $i \in P$.
    \end{itemize}
  \end{itemize}
  In this work, we will sometimes refer to games in normal form simply as \enquote{games}.
  This should not lead to confusion as we do not consider any other forms.
\end{definition}

Note that we only consider pure strategies, meaning that each player must choose one strategy deterministically, rather than choosing a probability distribution over multiple strategies (usually called mixed strategies).

Every game in normal form can be captured by an $n$-dimensional matrix with one dimension for each player and one index for every strategy in the strategy set of that player.
Every cell of the matrix contains a vector of utilities for each player for a particular strategy profile.

In this work, we focus mainly on 2-player games which can be conveniently displayed as 2-dimensional matrices.
In this case, we usually refer to the players the \textit{row player} and the \textit{column player}.
\autoref{tab:prisoners-dilemma} shows an example of such a matrix.

\begin{table}
  \caption{Prisoner's dilemma---a standard example of a game in normal form.
  Two suspects (row- and column-player) are placed in solitary confinement, with no means of communicating with each other.
  Each suspect has two options: either cooperate with the other by remaining silent, or defect by testifying against the other.
  If both suspects remain silent, each of them will serve one year in prison.
  If one defects while the other remains silent, the other will serve three years.
  If both defect, each of them will serve two years.
  }
  \label{tab:prisoners-dilemma}
  \centering
  \begin{tabular}{|c|c|c|}
    \hline
              & cooperate & defect \\
    \hline
    cooperate & -1, -1    & -3, 0  \\
    \hline
    defect    & 0, -3     & -2, -2 \\
    \hline
  \end{tabular}
\end{table}

Now that we have defined games in normal form, we can introduce some more standard notation.

\begin{definition}
  For a strategy profile $\vecs = (s_1, \dots, s_n) \in S$ we denote $\vecs_{-i} = (s_1, \dots, s_{i-1}, s_{i+1}, \dots, s_n)$ the opponents' profile, i.e., the vector $\vecs$ with element $s_i$ skipped.
  We denote $S_{-i} = S_1 \times \dots \times S_{i-1} \times S_{i+1} \times \dots \times S_n$ the set of all opponents' profiles.
  For a strategy $s_i' \in S_i$ of player $i$ we denote $(s_i', \vecs_{-i}) = (s_1, \dots, s_{i-1}, s_i', s_{i+1}, \dots, s_n) \in S$ the strategy profile $\vecs$ with element $s_i$ replaced by $s_i'$.
  We denote  $u_i(s_i', \vecs_{-i}) = u_i((s_i', \vecs_{-i}))$.
\end{definition}

In Nashian game theory, we explore the best strategy for a player, supposing that he knew the strategies that all other players would choose.
This is captured by the notion of best response.

\begin{definition}[Nashian best response]
  For a player $i \in P$, we say a strategy $s_i^* \in S_i$ is a Nashian best response to a strategy profile $\vecs_{-i}$ of his opponents if, supposing that $\vecs_{-i}$ stays fixed, player $i$ has no motivation to change his strategy from $s_i^*$ to another strategy.
  Formally, we say $s_i^*$ is a (\textit{weak}) Nashian best response if it satisfies
  \[
    u_i(s_i^*, \vecs_{-i}) \ge u_i(s_i', \vecs_{-i})\ \forall s_i' \in S_i.
  \]
  We say $s_i^*$ is a \textit{strict} Nashian best response if it satisfies
  \[
    u_i(s_i^*, \vecs_{-i}) > u_i(s_i', \vecs_{-i})\ \forall s_i' \in S_i \setminus \{s_i^*\}.
  \]
  In this work, if we do not specify the type, we always refer to the \textit{weak} type.
  Note that every strict best response is also a weak best response.
\end{definition}

In the example of Prisoner's dilemma in \autoref{tab:prisoners-dilemma}, the strict Nashian best response for the row player to \textit{cooperate} is \textit{defect}, and the strict Nashian best response to \textit{defect} is also \textit{defect}.

Having defined a best response, it is quite natural to ask whether there exists a strategy profile in which each player's strategy is the best response to all other players' strategies.
This is formally called the Nash equilibrium.

\begin{definition}[Nash equilibrium]
  We say a strategy profile $\vecs = (s_1, \dots, s_n)$ is a (\textit{weak}) Nash equilibrium if for every player $i \in P$, strategy $s_i$ is a (weak) Nashian best response to $\vecs_{-i}$.
  Similarly, $\vecs$ is a \textit{strict} Nash equilibrium if for every player $i \in P$, strategy $s_i$ is a strict Nashian best response to $\vecs_{-i}$.
\end{definition}

If we consider mixed strategies, at least one Nash equilibrium always exists~\cite{Nash51}.
However, for pure strategies, it is not always the case.

In the case of Prisoner's Dilemma, the strategy profile (\textit{defect}, \textit{defect}) is the only Nash equilibrium (both strict and weak).
In some games, there are additional weak equilibiria that are not strict.
However, in games such as Prisoner's Dilemma, this can clearly never be the case because there are no two strategy profiles with the same outcome for some player.
We say Prisoner's Dilemma is a \textit{game without ties}.

\begin{definition}[Game without ties]
  We say a game $G = (P, S, \uu)$ is without ties if it satisfies
  \[
    \forall i \in P,\ \forall \vecs, \vect \in S: \vecs \ne \vect \implies u_i(\vecs) \ne u_i(\vect).
  \]
\end{definition}

The \textit{dilemma} in Prisoner's Dilemma is that mutual cooperation would yield a better outcome for both players.
However, from a self-interested perspective, it is not considered a rational choice to cooperate under the Nashian assumptions.
We call this \enquote{better} state (\textit{cooperate}, \textit{cooperate}) Pareto-optimal.

\begin{definition}[Pareto-optimality]
  For a game $G = (P, S, \uu)$, we say a strategy profile $s \in S$ \textit{Pareto-dominates} profile $s' \in S$ if
  \begin{itemize}
    \item no player gets a worse payoff with $s$ than with $s'$: $u_i(s) \ge u_i(s')\ \forall i \in P$, and
    \item at lest one player gets a strictly better payoff with $s$ then with $s'$: $\exists i \in P: u_i(s) > u_i(s')$.
  \end{itemize}
  A strategy profile $s \in S$ is \textit{Pareto-optimal} if there is no profile $s' \in S$ that Pareto-dominates $s$.
\end{definition}

In this work, we also discuss symmetric games.
Informally, these are games that look the same from each player's perspective.

\begin{definition}[Symmetric game]
	We say a game $G = (P, S, \uu)$ is symmetric if there is a common set of strategies $S_c$ shared by all players:
  $S_i = S_c\ \forall i \in P$,
  and there is a common utility function $u_c$ shared by all players:
  $u_i = u_c\ \forall i \in P$.
  The utility only depends on the set of opponents' strategies (not on who is playing them):
  \[
    u_c(s_i, \vecs_{-i}) = u_c(s_i, \ppi_{-i})\ \forall s_i \in S_c, \vecs_{-i} \in S_c^{n-1}, \ppi_{-i} \text{ is a permutation of } \vecs_{-i}.
  \]
\end{definition}

\section{Non-Nashian Game Theory}
Suppose that there is a game in which some player $i$ has some strategies $s_i$ and $s_j$, $s_i \ne s_j$ such that no matter what strategies the opponents choose, player $i$ will get a better payoff by playing $s_i$ compared to $s_j$.
If the player wants to maximize utility, it would be irrational to play $s_j$.
If we suppose all players are rational, we can eliminate strategies such as $s_j$ because we know they will never be part of an outcome.
For player $i$ to choose the best strategy, it may be useful to first discard all the strategy profiles that are not possible outcomes.
This reduces the number of options to compare and to choose from.
Moreover, after discarding some of the profiles, the whole process may be repeated; maybe it will again discard some impossible outcomes.

\begin{definition}[Minimax rationalizibility]
  Given a game $G = (P, S, \uu)$, we define $\mathcal{R}_0(G) = S$ and $\mathcal{R}_k(G) \subseteq \mathcal{R}_{k-1}(G)$ the set of strategy profiles that are not minimax-dominated in the $i$\textsuperscript{th} round:
  \[
    \mathcal{R}_k(G) = \left\{\vecs \in \mathcal{R}_{k-1}(G) \mid \forall i \in P: \max_{\vect_{-i} \in S_{-i}}u_i(s_i, \vect_{-i}) \ge \max_{t_i \in S_i}\min_{(t_i, \vect_{-i}) \in \mathcal{R}_{k-1}(G)}u_i(t_i, \vect_{-i})\right\}.
  \]
  We say a strategy profile $\vecs \in S$ is minimax rationalizable (MR) if it survives all rounds: $\vecs \in \mathcal{R}_k(G)\ \forall k \in \N$.
\end{definition}

Minimax rationalizibility is a useful concept but in practice it turns out not to be very selective: in many games, most (or even all) strategies are minimax rationalizable.
Individual rationality implements a similar idea but provides a more granular selection: individal strategy profiles are compared, instead of the whole strategies.

\begin{definition}[Individual rationality]
  For any game $G = (P, S, \uu)$ we say a strategy profile $\vecs = (s_1, \dots, s_n) \in S$ is \textit{not} individually rational if there is a player $i \in P$ who could, by choosing a different strategy $s_i' \ne s_i$, ensure a higher worst-case payoff (across all possible opponent profiles) than $u_i(\vecs)$.
  Otherwise, we say $\vecs$ is individually rational.
  Formally, a strategy profile $\vecs$ is individually rational (IR) if
  \[
    \forall i \in P: u_i(\vecs) \ge \max_{t_i\in S_i}\min_{\vect_{-i} \in S_{-i}}u_i(t_i, \vect_{-i}).
  \]
\end{definition}

\begin{observation}
  A Nash equilibrium is always individually rational.
\end{observation}

While a Nash equilibrium is an outcome that is always individually rational, it might not be Pareto-optimal (as in the case of Prisoner's dilemma).
A~natural question to ask is whether, under some special conditions, we could ensure that rational players will always end up with a Pareto-optimal strategy profile.
For example, it is known that playing Prisoner's dilemma repeatedly for a fixed number of times (with both players knowing the number) does not motivate rational players to change strategies: each player will defect in every round.
However, when the total number of rounds is not known to the players, defecting in each round may not be a dominant strategy anymore.
For indefinitely long games, rational players can sustain the cooperative outcome.
% https://www.degruyter.com/document/doi/10.1515/9781400882168-018/html
% https://msp.org/pjm/1960/10-2/pjm-v10-n2-p02-p.pdf

Another assumption that one can make in order to change the outcomes drastically is (1) Necessery Knowledge of Strategies and (2) Necessary Rationality, as described by Fourny \cite{Fourny20}.
Informally, this means that
\begin{enumerate}[label=(\arabic*)]
  \item All players can perfectly predict the strategies chosen by their opponents.
  In other words, they know the strategy profile that the game will reach, before it is even played.
  \item All players are rational in all possible worlds.
  In other words, whatever strategy they choose, it maximizes their own utility.
  If one player would have chosen a different strategy, all the others would have known and would still have acted rationally.
\end{enumerate}
Making these two assumptions together allows to define a new kind of equilibrium.
Similar to minimax rationalizibility, it is based on iteratively eliminating strategy profiles that cannot possibly be an outcome of the game.
First, it is easy to see that any strategy profile that is not individually rational cannot be an outcome of the game because there would be some player who would surely be better off with a different strategy (and all players know this because of Necessare Knowledge of Strategies).
Thus, in the first round of elimination, we can eliminate all strategy profiles that are not individually rational.
Now, since every player knows that only individually rational strategy profiles are possible outcomes, they do not actually need to consider the eliminated profiles anymore.
This naturally leads to another round of elimination, in which we only keep profiles that are individually rational with respect to the set of possible outcomes from the previous round.

\begin{definition}[Perfectly transparent equilibrium]
  For a given game $G = (P, S, \uu)$, we define the $k$\textsuperscript{th}-level-preempted strategy profile $\mathcal{S}_k \subseteq S$ for $k \in \N$ as follows.
  $\mathcal{S}_0(G) = S$.
  For $k \ge 1$, we say a strategy profile is $k$\textsuperscript{th}-level-preempted if it does not Pareto-dominate the maximin utility over all strategy profiles that are not $(k-1)$\textsuperscript{st}-level preempted.
  The set of strategy profiles that survived the $k$\textsuperscript{th} round of elimination is
  \[
    \mathcal{S}_k(G) = \left\{\vecs \in \mathcal{S}_{k-1}(G) \mid \forall i \in P: u_i(\vecs) \ge \max_{t_i \in S_i}\min_{(t_i, \vect_{-i}) \in \mathcal{S}_{k-1}(G)} u_i(t_i, t_{-i})\right\}.
  \]
  A perfectly transparent equilibrium (PTE) is a strategy profile $\vecs \in S$ that never gets eliminated in the preemtpion process.
  In other words, $\vecs \in S$ is a PTE if $\vecs \in \mathcal{S}_k(G)\ \forall k \in \N$.
\end{definition}

For any game with no ties, if a PTE exists, then it is unique and Pareto-optimal~\cite{Fourny20}.

\section{Web Development}
\section{Data Analysis}
